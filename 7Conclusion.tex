%% LyX 2.3.6.1 created this file.  For more info, see http://www.lyx.org/.
%% Do not edit unless you really know what you are doing.
\documentclass[12pt,preprint,3p]{elsarticle}
\usepackage[latin9]{inputenc}
\usepackage{amssymb}

\makeatletter
%%%%%%%%%%%%%%%%%%%%%%%%%%%%%% User specified LaTeX commands.
%%
%% Copyright 2007, 2008, 2009 Elsevier Ltd
%%
%% This file is part of the 'Elsarticle Bundle'.
%% ---------------------------------------------
%%
%% It may be distributed under the conditions of the LaTeX Project Public
%% License, either version 1.2 of this license or (at your option) any
%% later version.  The latest version of this license is in
%%    http://www.latex-project.org/lppl.txt
%% and version 1.2 or later is part of all distributions of LaTeX
%% version 1999/12/01 or later.
%%
%% The list of all files belonging to the 'Elsarticle Bundle' is
%% given in the file `manifest.txt'.
%%

%% Template article for Elsevier's document class `elsarticle'
%% with numbered style bibliographic references
%% SP 2008/03/01
%%
%%
%%
%% $Id: elsarticle-template-num.tex 4 2009-10-24 08:22:58Z rishi $
%%
%%


%% Use the option review to obtain double line spacing
%% \documentclass[preprint,review,12pt]{elsarticle}

%% Use the options 1p,twocolumn; 3p; 3p,twocolumn; 5p; or 5p,twocolumn
%% for a journal layout:
%% \documentclass[final,1p,times]{elsarticle}
%% \documentclass[final,1p,times,twocolumn]{elsarticle}
%% \documentclass[final,3p,times]{elsarticle}
%% \documentclass[final,3p,times,twocolumn]{elsarticle}
%% \documentclass[final,5p,times]{elsarticle}
%% \documentclass[final,5p,times,twocolumn]{elsarticle}

%% if you use PostScript figures in your article
%% use the graphics package for simple commands
%% \usepackage{graphics}
%% or use the graphicx package for more complicated commands
%% \usepackage{graphicx}
%% or use the epsfig package if you prefer to use the old commands
%% \usepackage{epsfig}

%% The amssymb package provides various useful mathematical symbols
%% The amsthm package provides extended theorem environments
%% \usepackage{amsthm}

%% The lineno packages adds line numbers. Start line numbering with
%% \begin{linenumbers}, end it with \end{linenumbers}. Or switch it on
%% for the whole article with \linenumbers after \end{frontmatter}.
%% \usepackage{lineno}

%% natbib.sty is loaded by default. However, natbib options can be
%% provided with \biboptions{...} command. Following options are
%% valid:

%%   round  -  round parentheses are used (default)
%%   square -  square brackets are used   [option]
%%   curly  -  curly braces are used      {option}
%%   angle  -  angle brackets are used    <option>
%%   semicolon  -  multiple citations separated by semi-colon
%%   colon  - same as semicolon, an earlier confusion
%%   comma  -  separated by comma
%%   numbers-  selects numerical citations
%%   super  -  numerical citations as superscripts
%%   sort   -  sorts multiple citations according to order in ref. list
%%   sort&compress   -  like sort, but also compresses numerical citations
%%   compress - compresses without sorting
%%
%% \biboptions{comma,round}

% \biboptions{}


%\journal{Nuclear Physics B}

\makeatother

\begin{document}

\section{Conclusion and Outlook}

Starting with the optimization of the fuel cell, it is evident that
a desired power density curve can be approximated very well by adjusting
the parameters through the optimization. Interestingly, different
parameter settings also lead to the same power-density graph. This
means that the solution is not clear here, which in turn could be
used for a specific desired parameter setting. For further investigations
it would be important to include the temperature generated by the
operation in the simulations, which was choosen as constant for this
work. The effect of this temperature on performance is desirable.

Next was the achievement of a Statistically uniform and isotropic
field studied in an electromagnetic reverbration cuboid chamber, with
a Vivaldi antenna as source. The field turbulence was provided by
a stirrer whose geometry plays a crucial role in field distribution.
In this work, the shape of the stirrer wings were optimized to achieve
the best possible field distribution. This enabled the desired values
to be achieved except for the real part of the field in the y-direction.
In order to further improve the field distribution values, the number
of stirrers was increased to three pieces. The wing shapes have also
been optimized here. The field in the y-direction could be improved,
but the field values in the other spatial directions deteriorated.
Despite this fact, in all spatial directions, the desired acceptance
limits for the field distribution was observed. Next, a certain number
of two different geometric objects (cone and sphere) were attached
to the wall chamber and their influence on the field distribution
was examined. Here the hemispheres provided a better field distribution
than cones. Here the hemispheres provided a better field distribution
than cones, so this would be a better choice for further investigation.
For further investigations it would also be interesting to optimize
the topology of the stirrer. To do this, one does not assume a rectangular
sheet metal as the basic shape of a stirrer, but calculates the optimal
shape completely mathematically. Studying other geometric objects
and their shape and number that could be placed on the wall of the
chamber would also be of scientific importance.
\end{document}
